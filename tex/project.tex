\documentclass[a4paper,12pt,titlepage]{article}

% custom paragraph to have line break after using mbox to hack :-/
\newcommand{\myparagraph}[1]{\paragraph{#1}\mbox{}\\}

% import graphicx package so we can import images
\usepackage{graphicx}

% so we can use sidewaystable
% \usepackage[figuresleft]{rotating}

\begin{document}

\iffalse
Page 1: Title Page – including the title of the project, the name of the author, the
date, the word count and the statement specified below under report details.
\fi

\begin{titlepage}
    \begin{center}

        { \large Sommelier: A Recommender System}\\[0.4cm]

        { By Peter Chamberlin}

        { \today }

        \vfill

        { 7500 words. }

        { BSc Information Systems and Management Project Report, \\
          Birkbeck College, University of London. }

        \vspace{0.4cm}

        \textit{ This report is the result of my own work except where explicitly stated in the text. The report may be freely copied and distributed provided the source is explicitly acknowledged. }

    \end{center}
\end{titlepage}
\date{\today}




\iffalse
Page 2: Abstract - One page that summarises the report and the main findings or
results.
\fi

\begin{abstract}
    Recommender systems are prevalent on the Web, and an important selling point for businesses such as Amazon and Netflix, who trade on the quality of their recommendations to users. Academic research, thanks partly to exceptional interest in the field precipitated by the Netflix Prize, has tended to concentrate research in the field of movie recommendations.
    Service oriented architectures have grown in popularity on the Web in recent years. Such architectures lend themselves to scalability and maintainability, and in particular satisfy the needs of organizations who deliver the same data and content to diverse set of clients, such as websites and mobile apps.
    In this project I implement a recommender system as a web service for wine recommendations. Using as a starting point the wines, authors and ratings data from wine website Decanter.com I implement a web API serving JSON data using the Python language, MySQL, the Flask framework, and a number of Python libraries.
    Implementing a primitive collaborative filtering system but finding my source data to be exceptionally sparse for recommendation, I explore and implement imputation techniques using matrix factorization and singular value decomposition in an attempt to boost the system's ability to make good recommendations.
    I find that both matrix factorization and singular value decomposition significantly improve the recommendation quality of the system, but due to time contraints I am forced to leave some avenues of inquiry unexplored.
\end{abstract}



\iffalse
Page 3: Table of Contents - including page numbers of each chapter heading and
each appendix.
\fi

\tableofcontents

\iffalse 
Chapter 1: Introduction - the topic, the background, why the topic is relevant or of interest to you, what you hoped to achieve, the aims and objectives of the project.  
\fi

\section{Introduction}

\paragraph{Recommender Systems.}

Since their origin in the mid-1990s with systems such as Tapestry \cite{Goldberg92} and GroupLens \cite{Resnick94}, recommender systems have become ubiquitous on the World Wide Web, being employed by some of the worlds largest online businesses as core parts of their offering to users.

Companies such as Amazon, Netflix, Facebook and Twitter use recommender systems to make all manner of suggestions to their users. These recommendations include such things as products, movies, news stories and other interesting users. 

It is the growth of the Web, which is now ubiquitous itself, that has given companies the ability to draw on unprecedented amounts of data about their users' preferences. At the same time the Web has made it easier than ever to reach their users with tailored suggestions.

Amazon's system of product recommendations using item-to-item collaborative filtering is regarded as a ``killer feature''\cite{Fortune12}, and is one of the defining features of the Amazon brand experience. Amazon state their mission to be, ``to delight our customers by allowing them to serendipitously discover great products''\cite{Fortune12}.

Netflix's movie recommender system \emph{Cinematch}  ``Netflix Prize'' competition 


\paragraph{Social Recommender Systems.}

Guy et. al. \cite{Guy11} describe social recommender systems as, ``a class of recommender systems that target the social media domain''. The social media domain 

\paragraph{Recommending Wines.}

\paragraph{Service Oriented Architecture.}

\paragraph{Aims and Objectives.}

I aim to produce a recommender system for wines which takes advantage of both ratings and tasting notes to recommend both interesting wines and interesting users to users of the system.

Rather than implement a full graphical interface for the system I have chosen to develop an HTTP API.

In doing so I will explore the field of recommender systems, 

What is a social recommender system?

Why are these systems interesting - Benefits - Challenges

Typical applications...  - Movies - Products (i.e. Amazon)

Applications in wine domain - what's the same - what's different

What will this project do?  - implement a recommender system for wines -
exploration of techniques etc.



\iffalse
Chapter 2: Literature Review and Context - the setting of the project in the context of other relevant work or theories or results. How this setting influenced the project.
\fi

\section{Literature Review}\label{literature review}

The term \textit{recommender system} was coined by Resnick and Varian \cite{Resnick97} to describe  a system that ``assists and augments'' the ``natural social process'' of recommendation, with Resnick and Varian stating that they preferred it to the more narrow term ``collaborative filtering'' used by Goldberg et al. \cite{Goldberg92}  to describe their Tapestry system.

The growth of the World Wide Web has seen recommender systems become a ubiquitous part of everyday life, with companies such as Amazon, Facebook, Twitter and Google using making recommendations to millions of us every day.

\ldots

There are several main categories of filtering technique employed in recommender systems. Burke \cite{Burke02} presents five: \textit{collaborative}, \textit{content-based}, \textit{demographic}, \textit{utility-based} and \textit{knowledge-based}. Table \ref{table:burke02} details these methods and their properties.

\begin{table}[ht]
    \caption{Recommendation Techniques, reproduced from Burke, 2002 \cite{Burke02}}
    \centering
    \begin{tabular}{p{2.5cm} p{3.5cm} p{3.5cm} p{3.5cm}}
        Technique & Backgroud & Input & Process
        \\\hline\hline
        Collaborative & Ratings from \textit{U} of items in \textit{I}. & Ratings from \textit{u} of items in \textit{I}. & Identify users in \textit{U} similar to \textit{u}, and extrapolate from their ratings of \textit{i}. \\
        Content-based & Features of items in \textit{I}. & \textit{u}'s ratings of items in \textit{I}. & Generate a classifier that fits \textit{u}'s rating behaviour and use it on \textit{i}. \\ 
        Demographic & Demographic information about \textit{U} and their ratings of items in \textit{I}. & Demographic information about \textit{u}. & Identify users that are demographically similar to \textit{u}, and extrapolate from their ratings of \textit{i}. \\
        Utility-based & Features of items in \textit{I}. & A utility function over items in \textit{I} that describes \textit{u}'s preferences. & Apply the function to the items and determine \textit{i}'s rank. \\
        Knowledge-based & Features of items in \textit{I}. Knowledge of how these items meet a user's needs. & A description of \textit{u}'s needs or interests. & Infer a match between \textit{i} and \textit{u}'s need. \\
        \\\hline
    \end{tabular}
    \label{table:burke02}
\end{table}

The properties by which Burke, 2002 \cite{Burke02}, classifies these systems are \textit{background data}, which exists prior to recommendation, \textit{input data}, the data that has been contributed by the user, and \textit{process}, the method by which recommendations are arrived at by using the \textit{background data} and \textit{input data}.

Burke, 2002 \cite{Burke02}, asserts that \textit{collaborative filtering} is probably the most widely used and mature of these types, citing \textit{GroupLens} \cite{Resnick94} and \textit{Tapestry} \cite{Goldberg92} as important examples of such systems, as well as several others. 

\textit{Collaborative filtering} a process whereby a user's preferences for items are inferred by the comparison of their previous preferences with the preferences of others. In itself this is a fairly straighforward principle, the comparison of two vectors of items and ratings or preferences, but there are a plurality of approaches. Burke, 2002 \cite{Burke02}, 

\textit{Content-based}, like \textit{collaborative filtering}, builds up a long term profile of a user's interests and preferences \cite{Burke02}.



\myparagraph{Hybrid Systems.}




--- off the web (?)
--- on the web...

-- What are the methods employed in recommender systems?

-- Collaborative Filtering
  - User-based filtering
  - Item-based filtering

-- Content-Based Filtering
  - Variants

-- Characteristics of the domain

--- Cold start problem
--- Sparsity problem
--- ... etc.



\section{Method}\label{method}

\iffalse
the overall approach and rationale.
Why the project was tackled in the chosen way, and why other ways were ruled out.
\fi

\subsection{Methodology}

Given the exploratory nature of this project I elected to take an incremental and iterative approach, developing small parts of the system at any time (increments), and iterating over those parts with improvements as necessity dicated and time allowed. This approach is based on that laid out by Cockburn (2008 \cite{Cockburn08}).

\subsubsection{Phases}

The main phases of development would be:

\begin{enumerate}
    \item Clean up and migrate data
    \item Create initial API app
    \item Connect API with database
    \item Implement routes for API access to wines and authors
    \item Augment API routes for wines and authors with recommendations
    \item Iterate on recommendation methods, evaluating and improving quality
\end{enumerate}

My reason for taking this approach, rather than following a formal development methodology such as the waterfall model, was that my aims and objectives were unbounded, it the sense that there would not be a point at which my system was complete, only a point at which it was minimally complete, followed by a succession of points at which is was improved.

\subsubsection{Minimum Viable Product}

It was clear that I would be doing a large amount of experimental programming, and given my lack of prior experience in the problem domain I felt it inappropriate to attempt to quantify my expectations for the system in terms of detailed requirements. Nevertheless there were very clear minimum objectives for the system, without which it would not be possible to claim any degree of success.

The system should at least:

\begin{itemize}
    \item Provide an HTTP API for accessing wine and user information from the Decanter.com tastings database.
    \item Augment the API results for wines and authors with appropriate recommendations of other similar or interesting wines and author.
    \item Provide API results suitable for machine interpretation by web or mobile applications.
    \item Provide a mechanism by which to evaluate recommendation quality.
\end{itemize}

These requirements in the least should be fulfilled by the system. With this having been done the focus of the project will be on maximising the quality of recommendations.

\subsection{Technologies and Tools}

\subsubsection{Python}

As the main progamming language for my project I chose to use Python. There were several candidate languages, not least Java, but I decided on Python because it has a number of attributes which lent themselves particularly to this project:

\begin{itemize}
    \item Extensive mathematical and scientific libraries, such as numpy \cite{Numpy} and scipy \cite{Scipy}.
    \item Extensive detailed documentation \cite{PythonDocs}.
    \item Widely used in web development, such as by Google and YouTube \cite{PythonQuotes}.
    \item Interactive interpreter, allowing command line interaction and supporting scripting on Unix-like systems \cite{PythonInterpreter}.
\end{itemize}

One deciding factor was that my first enquiry into recommender systems was reading Segaran's code examples in Chapter 2 of Collective Intelligence (2007, Ch.2 \cite{Segaran07}), where the language he uses for his code examples is Python. 

In addition to its suitability for tasks around recommender systems, Python has a solid heritage of web application frameworks, such as Django \cite{DjangoProject} and Flask \cite{Flask}. Django is a fully featured website building framework, and as such carries many features unnecessary for my project, whereas Flask, a ``micro-framework'' \cite{Flask}, appeared to be more lightweight and simple to implement. Therefore I chose to implement my API using Flask \cite{Flask}.

For the most part I considered that my system would suit the stateless, non-persistent nature of a Python web application. The only concern in this regard would be that I would need to recreate objects in memory from scratch with each request rather than persist them as I might using another language, such as using Java with the JPA\cite{JavaJPA}. It was reasonable to suppose that in generating recommendations I would potentially be creating large objects in memory, and that there may be a performance deficit incurred by having to rebuild such object on a per request basis. I resolved that should the lack of persistence prove problematic down the line I would be able to use a persistence mechanism such as Memcached \cite{PythonMemcached} to serve this purpose, and found that there is wide support for such a solution using Python and Flask \cite{FlaskMemcached}.

\subsubsection{MySQL}

Originally I had envisaged a system backed by a PostgreSQL \cite{PostgreSQL} RDBMS, but having received the Decanter.com data as a MySQL \cite{MySQL} database it did not seem, comparing the two systems, that there would be any significant benefit migrating the data to PostgreSQL. Both are widely used in production, and have similar feature sets. For a short time I considered using a NoSQL database such as MongoDB \cite{MongoDB} for my project, but decided against such a solution, recognising that such document-oriented systems are not ideal when joining between tables in the way that I would need to for my wines and tasting notes. It seemed that an RDBMS was ideally suited to the purpose, and there was no reason why that shouldn't be MySQL.

\subsubsection{GitHub}

Given the iterative nature of my development process I envisaged a need to be able to easily version my source code, possibly running several different versions at once, with the ability to revert changes back to any previous state. I also wanted a remote backup of my system in case of problems with my own development computer. In order to do be able to do these things I chose to store my code as a project in GitHub \cite{GitHub}, which is a web service providing Git version control. I chose to use GitHub for my notes and project files also, so that my entire project was stored, versioned and backed up together.



\section{The Sommelier System}\label{design}

\myparagraph{Data Cleanup}

The data source I have used for my project is the wines database belonging to Decanter.com\cite{DecanterCom}. The database contains nearly 40,000 professional ratings and tasting notes for wines from as far back as 1986, featuring vintages as far back as 1917.

The original database is highly inconsistent, displaying a mixture of design approaches and a variable quality of data. This is consistent with the fact that the database has been developed over a long period of time by a number of different developers with varying levels of skill, and that wine journalists making entries into the database have taken a number of idiosyncratic approaches to data entry.

Nevertheless I considered there to be a great deal of useful and interesting information in the database, with it to contain usable ratings and/or tastings for over 33,000 wines.

\begin{figure}[h!]
    \caption{Decanter.com Database}
    \centering
        \includegraphics[width=14cm]{DecanterWineDB}
    \label{fig:decanterdb}
\end{figure}

The WineInfo table is a mixture of foreign keys joining to very small tables, such as WineInfo.type\_id joining to WineType.id where WineType is a table with only two attributes. This approach, stiving for a high degree of normalisation, contrasts with the fact that the same table also has the attribute second\_wine, as a string which only holds data in 450 of the 38762 entries in the table.

\myparagraph{Creating The Sommelier Dataset}

\begin{figure}[h!]
    \caption{Sommelier Database}
    \centering
        \includegraphics[width=10cm]{SommelierDBSimple}
    \label{fig:sommelierdb}
\end{figure}

For the new Sommelier database, Figure \ref{fig:sommelierdb}, I decided to denormalise [explanation/citation needed!] the wine data. This enables the data to be queried without joins, maximising the simplicity and execution speed of the queries [citation needed]. Denormalization makes data integrity difficult to maintain however, as there are potentially a large number of records to update for any change in a duplicated value. In this case an appellation or sub-region name changing might require thousands of records to be amended. Creating and editing wines is not a requirement of my system though, so for the purposes of this project the wine and tasting data is static and will not be subject to updates. For this reason the duplication of data within the Wine records is not problematic. In a real world setting this would need to be revisited.

Much of the data from the original database was disregarded entirely.

The tables WineStyleNarrow and WineStyleBroad contained generic text descriptions for wine (``rich and creamy'', ``crisp and tangy'' etc.). I initially considered this to have potential for migration into tag data which I could reuse as part of my filtering. Unfortunately less than 6435 of the records in WineInfo had non-null values for their style\_narrow\_id field, and only 3397 of these had corresponding records in the Tasting table. This figure was only around 10\% of the number of wines I expected the Sommelier database to contain so I decided that the WineStyle* tables were probably not worthwhile to migrate.

The WineType table was ignored because no wines corresponded to it; no WineInfo.type\_id record matched any WineType.id.

\myparagraph{The Author Problem}

The biggest shortcoming of the dataset is that the author of a tasting note is often not recorded. The number of wines with notes and known authors is only 1401, with there being 18 named authors on the system. 

Table \ref{table:authors} shows the distribution of tastings amongst authors, only 5 of which have tasted and rated more than 100 wines in the database.

\begin{table}[ht]
\caption{Authors of tasting notes and ratings}
\centering
\begin{tabular}{c c}
\\\hline\hline
Author               & Wines tasted, with notes and rating
\\\hline
Amy Wislocki         &            28 \\
Andrew Jefford       &           105 \\
Beverley Blanning MW &            13 \\
Carolyn Holmes       &             1 \\
Christelle Guibert   &           119 \\
Clive Coates MW      &             6 \\
David Peppercorn     &            44 \\
Gerald D Boyd        &             7 \\
Harriet Waugh        &           250 \\
James Lawther MW     &           226 \\
John Radford         &             2 \\
Josephine Butchart   &            24 \\
Norm Roby            &             4 \\
Rosemary George MW   &             6 \\
Serena Sutcliffe     &            31 \\
Stephen Brook        &            19 \\
Steven Spurrier      &           497 \\
\\\hline
\end{tabular}
\label{table:authors}
\end{table}

In some cases an author's initials or full name are recorded within the text of a tasting note. I decided that extracting and making use of these was impractical given the time constraints of this project.

DESCRIBE DATA SETS BEFORE AND AFTER

THE SOMMELIER DATASET

Having analysed the dataset and conceived an ideal schema, I needed to decide what the criteria to apply when extracting my new dataset from the source data.

Given that the purpose of the dataset is social recommendations, the first decision I made was to discard any wines without both tasting notes and a rating, whether.


\section{Testing and Evaluation}\label{testing}

\subsection{Testing the Code}

\subsubsection{Unit Tests}

For the main components of the system I have produced unit tests (see Appendix \ref{app:unittests}) which are written using Python's standard, JUniti-inspired test framework, unittest (unittest framework, 2013 \cite{PythonUnittest}). These tests cover the classes Sommelier, SommelierBroker, SommelierRecommenderBase, and various parts of recommender classes SommelierPearsonCFRecommender and SommelierYeungMFRecommender.

In each case I have attempted to ensure that I test key units of work, using the python's mock library to factor out dependencies. This way the unit tests for one class will not fail because of a problem in another class, making maintanance of both the tests and the application code easier.

I have used the nose module as a test runner (Testing with nose, 2013 \cite{ReadthedocsNose}), which automatically locates and executes test files with the project. Automatically discovering and running all tests with nose is trivial:

\footnotesize
\begin{verbatim}
cd /path/to/sommelier
./flask/bin/nosetests
\end{verbatim}
\normalsize

\subsubsection{Functional Testing}

\begin{figure}[h!]
    \caption{Testing the Sommelier API using Postman}
    \centering
        \includegraphics[width=15cm]{Postman}
    \label{fig:postman}
\end{figure}

I have not implemented any automated functional testing, instead relying on manual testing to ensure that the API functions as it should. For this purpose I have used Postman (Postman website, 2013 \cite{Postman}), which is an extension for the Google Chrome web browser (Chrome Extension \cite{ChromePostman}) designed as a testing client for REST APIs. Figure \ref{fig:postman} shows the interface for Postman in Google Chrome.

\subsection{Testing Recommendations}

\begin{figure}[h!]
    \caption{Testing the SommelierYeungMFRecommender with Movielens Data}
    \centering
        \includegraphics[width=15cm]{MAE_Yeung_ML100k}
    \label{fig:movielenstest}
\end{figure}

When I came to test my recommendations, the most developed recommender implementation in my system was the SommelierYeungMFRecommender, based on a matrix factorization algorithm published on a blog by Albert Yeung (Quuxlabs, Matrix Factorization: A Simple Tutorial and Implementation in Python, 2010 \cite{Yeung10}). This system imputes a matrix of predicted values by using a gradient descent algorithm to iteratively improved its similarity to the original data.

Testing recommendations made by the system has been challenging. Of the three metrics put forward by Shani and Gunawardana (2011 \cite{Shani11}) that I raised in the literature review: prediction accuracy, item-space coverage and user-space coverage, I was only succesful in attempting to test one: prediction accuracy.

The metric I prefered for the purpose of evaluation was Mean Absolute Error (MAE), which is a commonly used measure (Su and Khoshgoftaar, 2009 \cite{Su09}).

I wrote a method, split\_data\_evaluation(), in the class SommelierYeungMFRecommender. This splits the tastings into two sets according to a percentage split, one for testing and one for training. The training data used to impute a matrix of predicted ratings, including ratings for items in the test set which are unseen by the system. Once the imputation is complete the predicted ratings are compared with the true ratings in the test set and the MAE is calculated, along with the standard deviation of the error (see Appendix \ref{app:yeungmf}).

I did not have a great deal of time to carry out my tests, having estimated a week for this task. Unfortunately I underestimated the time cost of running many iterations of a computationally intensive process, finding that each iteration of tests and improvements took several hours. By the time I had a satisfactory testing system there was no more development time left available.

I did produce one good set of data supporting the effectiveness of Yeung's algorithm for collaborative filtering, which was conducted using test data from the Movielens 100k data set (GroupLens, MovieLens Data Sets, 2011 \cite{MovielensDatasets}). Figure \ref{fig:movielenstest} shows Yeung's algorithm successfully reducing the MAE of the MovieLens data as iterations increase. When casually compared (they are not directly comparable) to the MAE results reported by Su and Khoshgoftaar (Su and Khoshgoftaar, 2006 \cite{Su06}), where the best MAE was 0.769 for the tests they undertook, the MAE of just over 0.8 for the SommelierYeungMFRecommender over 16 iterations looks quite promising. To gather this data I implemented the function split\_data\_evaluate\_movielens\_file() on the SommelierYeungMFRecommender class.

I did not satisfactorily test the system against the Decanter.com database. The sparsity of that data does not necessarily preclude successful results, but it was not until late in the project that I recognised the correct way to evaluate the system. I already spent a certain amount of effort testing the Decanter.com data in an invalid manner. Implementing the train/test system with the MovieLens data was the last thing I had time to do, unfortunately. A few more days would have seen the results for the Decanter.com data.



\section{Conclusion}\label{conclusions}

In many ways this project can be considered successful. Some objectives have been achieved well, such as that I have developed an API for the Decanter.com wine tasting data, and augmented the responses with recommendations.

The components of the system are fairly decoupled. I have successfully abstracted away a great deal of complexity, so that between classes there is little data shared, and the system's components operate largely independently of each other. In general I am very happy with the system's architecture.

It was always my goal, however, to be able to assert that the recommendations are good, which I am not able to do. Additionally, I have not succeeded in exploiting anything specific to wines when making my recommendations. The recommender I have produced is no different to one that might produce movie recommendations, or recommendations for any other generic rated item.

On balance I don't think that I managed my time as well as I could, and the project is less successful as a result.



\section{Review}\label{review}
Review / reflections of the project on a personal level. What has been achieved? What were the problems, and how were they overcome?

Lessons learnt\ldots
 - Data cleanup very time consuming
 - Literature vast -> plural techniques for recommendation: very difficult to work out what strategy is best for given situation.



\begin{thebibliography}{9}

    \bibitem{Burke02} Burke, \emph{Hybrid Recommender Systems: Survey and Experiments}, User Modeling and User-Adapted Interaction, Volume 12 Issue 4, November 2002, Pages 331 - 370. Kluwer Academic Publishers: Hingham, MA, USA

    \bibitem{Debnath08} Debnath, Souvik and Ganguly, Niloy and Mitra, Pabitra, \emph{
        Feature weighting in content based recommendation system using social network analysis}, Proceedings of the 17th international conference on World Wide Web, WWW '08, 2008, Beijing, China, Pages 1041 - 1042. ACM: New York, NY, USA,

    \bibitem{Goldberg92} Goldberg, D. Nichols, D., Oki, B. M., and Terry, D., \emph{Using collaborative filtering to weave an information tapestry}, Commun. ACM 35, 12 (Dec. 1992), 61--70.

    \bibitem{Fortune12} Mangalindan, J. P., \emph{Amazon's Recommendation Secret}, July 2012. URL: http://tech.fortune.cnn.com/2012/07/30/amazon-5/

    \bibitem{Resnick94} Resnick, P., Iacovou, N., Sushak, M., Bergstrom, P., Riedl, J., \emph{GroupLens: An open architecture for collaborative filtering of netnews}, 1994 ACM Conference on Computer Supported Collaborative Work, 1994. Association of Computing Machinery, Chapel Hill, NC.

    \bibitem{Resnick97} Resnick, P., Varian, H. R., \emph{Recommender Systems}, 1997. Communications of the ACM, 40 (3), 56-58. Association of Computing Machinery, Chapel Hill, NC.


\iffalse
A. Y. Ng and M. I. Jordan. On discriminative vs generative
classifiers: A comparison of logistic regression and naive
bayes. In Neural Information Processing Systems, pages
841–848, Vancouver, Canada, december 2001. MIT Press.
2
Robles, V.; Larranaga, P.; Menasalvas, E.; Perez, M.S.; Herves, V.; , "Improvement of naive Bayes collaborative filtering using interval estimation," Web Intelligence, 2003. WI 2003. Proceedings. IEEE/WIC International Conference on , vol., no., pp. 168- 174, 13-17 Oct. 2003
doi: 10.1109/WI.2003.1241189
keywords: {Algorithm design and analysis;Clustering algorithms;Collaboration;Collaborative work;Data mining;Filtering algorithms;Probability;Recommender systems;Scalability;Training data; Bayes methods; Web sites; groupware; information filters; learning (artificial intelligence); statistical analysis; Bayesian classifier; UCl repository; Web data; collaborative filtering; ecommerce site; interval estimation; naive Bayes method; recommender system; semi naive Bayes method;}
URL: http://ieeexplore.ieee.org/stamp/stamp.jsp?tp=&arnumber=1241189&isnumber=27823

Xiaoyuan Su; Greiner, R.; Khoshgoftaar, T.M.; Xingquan Zhu; , "Hybrid Collaborative Filtering Algorithms Using a Mixture of Experts," Web Intelligence, IEEE/WIC/ACM International Conference on , vol., no., pp.645-649, 2-5 Nov. 2007
doi: 10.1109/WI.2007.10
keywords: {Clustering algorithms;Collaborative work;Computer science;Filtering algorithms;International collaboration;Motion pictures;Niobium;Predictive models;Recommender systems;USA Councils;information filtering;content-boosted CF;experts mixture;hybrid collaborative filtering algorithms;memory-based algorithms;pure content-based CF algorithms;pure model-based algorithms;sequential mixture CF;}
URL: http://ieeexplore.ieee.org/stamp/stamp.jsp?tp=&arnumber=4427165&isnumber=4427044

Xiaoyuan Su; Taghi M. Khoshgoftaar; , "Collaborative Filtering for Multi-class Data Using Belief Nets Algorithms," Tools with Artificial Intelligence, 2006. ICTAI '06. 18th IEEE International Conference on , vol., no., pp.497-504, Nov. 2006
doi: 10.1109/ICTAI.2006.41
keywords: {Bayesian methods;Collaboration;Collaborative work;Filtering algorithms;Logistics;Predictive models;Recommender systems;Regression tree analysis;Robustness;Scalability;belief networks;data handling;groupware;information filtering;Bayesian belief nets;Pearson correlation-based collaborative filtering;data sparseness;extended logistic regression;multiclass collaborative filtering data;recommender system;tree augmented naive Bayes model;}
URL: http://ieeexplore.ieee.org/stamp/stamp.jsp?tp=&arnumber=4031936&isnumber=4031859



BibTex:

@conference {236,
title = {GroupLens: An open architecture for collaborative filtering of netnews},
booktitle = {1994 ACM Conference on Computer Supported Collaborative Work Conference},
year = {1994},
month = {10/1994},
pages = {175-186},
publisher = {Association of Computing Machinery},
organization = {Association of Computing Machinery},
address = {Chapel Hill, NC},
abstract = {<p class=``abstract''>Collaborative filters help people make choices based on the opinions of other people. GroupLens is a system for collaborative filtering of netnews, to help people find articles they will like in the huge stream of available articles. News reader clients display predicted scores and make it easy for users to rate articles after they read them. Rating servers, called Better Bit Bureaus, gather and disseminate the ratings. The rating servers predict scores based on the heuristic that people who agreed in the past will probably agree again. Users can protect their privacy by entering ratings under pseudonyms, without reducing the effectiveness of the score prediction. The entire architecture is open: alternative software for news clients and Better Bit Bureaus can be developed independently and can interoperate with the components we have developed.</p>},
doi = {http://doi.acm.org/10.1145/192844.192905},
author = {Resnick, P. and Iacovou, N. and Sushak, M. and Bergstrom, P. and J. Riedl}
}


\fi

\end{thebibliography}



\end{document}
Chapter 3: Research/Development Method - the overall approach and rationale.
Why the project was tackled in the chosen way, and why other ways were ruled out.
Chapter 4: Data/Findings/Designs - the project outcome. This might be data
collected and tabulated or the design of a program, or whatever outcome was
obtained.
Chapter 5: Analysis/Evaluation/Testing – assessing or testing the project outcome.
If the project is of type 2 are the results plausible? If the project is of type 3 or 4 then
any computer code should be tested using a range of inputs.
Chapter 6: Conclusions/Recommendations - as a result of the project. The project
does not need to have a positive conclusion. For example, it might prove that some
system was not effective or successful. You should indicate to what extent your
objectives have been achieved.
Chapter 7: Review/Reflections - this is often missed out by students but is very
important. It is an opportunity to, firstly, review on a personal level what you have
achieved, how you achieved it, what took the most time, the problems faced, the way
in which they were overcome, etc. Secondly, it is an opportunity to reflect on the
project with the benefit of hindsight. What might have been done differently? Was the
research method adequate? How could the project have been more successful?
Examiners like to see evidence of learning and mature reflection
Chapter 8: References - all references should be cited in the body of the report. A
typical reference in the report might take the form, “Donar and Kebab (1996) suggest
that high cholesterol levels do not lead to heart disease\ldots.” or “empirical eating studies
show that\ldots. (Donar and Kebab, 1996)”. The full title of the article or book or web
page in which Donar and Kebab make these assertions is then given in the list of
references. Where possible, use an article or a book rather than a web page. The idea
of references is not just to substantiate statements and arguments but also to make it
possible for other people to find the references. Normally, for a book, you should list
author(s), title, publisher, date of publication, relevant page number(s). It can be
difficult to locate the relevant part of a book if the page numbers are omitted. For an
article list author(s), title of article, name of journal, volume and issue number, date,
and page numbers of the article. In the academic world references are regarded as
very important and poor referencing will certainly detract from the project report. Do
not under any circumstances quote from a source without making it clear that
you are quoting. Any quote must be accompanied by an appropriate reference.
Chapter 9: Bibliography - list any relevant literature that has not been cited in the
report. (It is not a very well-kept secret that examiners tend to think that anything in a
bibliography has not in fact been read by the student. Of course this is a monstrous
slur but nevertheless do not waste too much time on the bibliography. Concentrate on
the references!)
Chapter 10: Appendices - these are not obligatory. Only put in relevant items not
already in the body of the report. These might include a questionnaire used to gather
information, a list of the people interviewed and their companies, transcripts of
interviews, detailed data, program listings, test results, etc. Any appendix should be
referred to in the main part of the report and not just stuck at the end of the report
without explanation. It is very important that an examiner can find evidence for the
claims you make in your report. The appendices are the place to put such evidence
without cluttering up the main part of the report.

