\iffalse
Chapter 2: Literature Review and Context - the setting of the project in the context of other relevant work or theories or results. How this setting influenced the project.
\fi

\section{Literature Review}\label{literature review}

The term ``recommender system'' was coined by Resnick and Varian \cite{Resnick97} to describe  a system that ``assists and augments'' the ``natural social process'' of recommendation, preferring it to the more narrow term ``collaborative filtering'' used by Goldberg et al. \cite{Goldberg92}  to describe their Tapestry system. 

There are now several main categories of filtering technique employed in recommender systems. Burke \cite{Burke02} looks at five of these: collaborative, content-based, demographic, utiliy-based and knowledge-based (Burke, 2002). 

\myparagraph{Collaborative Filtering.}




--- off the web (?)
--- on the web...

-- What are the methods employed in recommender systems?

-- Collaborative Filtering
  - User-based filtering
  - Item-based filtering

-- Content-Based Filtering
  - Variants

-- Characteristics of the domain

--- Cold start problem
--- Sparsity problem
--- ... etc.

